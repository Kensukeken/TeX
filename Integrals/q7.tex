\documentclass{article}
\usepackage{amsmath}
\usepackage{amssymb}

\begin{document}

We can write the given integral as:

\begin{align*}
\int_0^\infty \left(x-\frac{x^2}{2}+\frac{x^3}{2\cdot4\cdot6}-\cdots\right)\left(1+\frac{x^2}{2^2}+\frac{x^2}{2^2\cdot4^2}+\cdots\right) dx
\end{align*}

Let's simplify the first factor in the integrand by writing it as an infinite series:

\begin{align*}
x - \frac{x^2}{2} + \frac{x^3}{2\cdot4\cdot6} - \cdots &= \sum_{n=1}^\infty (-1)^{n+1} \frac{x^n}{n\cdot(n!)^2} \
&= \sum_{n=1}^\infty \frac{(-1)^{n+1}}{n} \cdot \frac{x^n}{(n!)^2}
\end{align*}

Next, let's simplify the second factor in the integrand by writing it as an infinite product:

\begin{align*}
1 + \frac{x^2}{2^2} + \frac{x^2}{2^2\cdot4^2} + \cdots &= \prod_{n=1}^\infty \left(1 + \frac{x^2}{(2n)^2}\right) \
= \frac{\sinh x}{x}
\end{align*}

where we have used the identity $\sinh x = x \cdot \prod_{n=1}^\infty \left(1 + \frac{x^2}{(2n)^2}\right)$.

Therefore, the given integral can be written as:

\begin{align*}
\int_0^\infty \left(x-\frac{x^2}{2}+\frac{x^3}{2\cdot4\cdot6}-\cdots\right)\left(1+\frac{x^2}{2^2}+\frac{x^2}{2^2\cdot4^2}+\cdots\right) dx\\\ = \int_0^\infty \sum_{n=1}^\infty \frac{(-1)^{n+1}}{n} \cdot \frac{x^n}{(n!)^2} \cdot \frac{\sinh x}{x} dx \\\\
= \sum_{n=1}^\infty \frac{(-1)^{n+1}}{n} \cdot \frac{1}{(n!)^2} \int_0^\infty x^n \sinh x , dx \\\
= \sum_{n=1}^\infty \frac{(-1)^{n+1}}{n} \cdot \frac{1}{(n!)^2} \cdot 2^n n! \\\
= \sum_{n=1}^\infty \frac{(-1)^{n+1}}{n} \cdot \frac{2^n}{(n!)^1} \
= \sum_{n=1}^\infty (-1)^{n+1} \frac{(2/3)^n}{n}
\end{align*}

where we have used the identity $\int_0^\infty x^n \sinh x , dx = 2^n n!$.

The last series can be recognized as the Taylor series expansion of $\ln(1+x)$ evaluated at $x=-2/3$, so we have:
\begin{align*}
\sum_{n=1}^\infty (-1)^{n+1} \frac{(2/3)^n}{n} &= \ln\left(1-\frac{2}{3}\right) \
= \ln\left(\frac{1}{3}\right) \
= -\ln 3
\end{align*}

Therefore, the value of the given integral is $-\ln 3$.
\end{document}