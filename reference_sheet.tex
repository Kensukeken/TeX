\documentclass[10pt]{article}
\usepackage[margin=0.5in]{geometry}
\usepackage{multicol}
\usepackage{xcolor}
\definecolor{titlecolor}{RGB}{0,0,128}
\usepackage{titlesec}
\usepackage{enumitem}
\usepackage{listings}
\usepackage{tcolorbox}
\setlength{\columnsep}{0.3in}
\pagestyle{empty}

% Define custom colors
\definecolor{codebg}{rgb}{0.97,0.97,1.0} 
\definecolor{keywordcolor}{rgb}{0.4,0.1,0.6} 
\definecolor{commentcolor}{rgb}{0.6, 0.4, 0.8} 
\definecolor{stringcolor}{rgb}{0.1,0.3,0.6}   

% Listings setup
\lstset{
  backgroundcolor=\color{codebg},
  basicstyle=\ttfamily\footnotesize,
  breaklines=true,
  frame=single,
  showstringspaces=false,
  tabsize=2,
  language=Java,
  keywordstyle=\color{keywordcolor}\bfseries,
  commentstyle=\color{commentcolor}\itshape,
  stringstyle=\color{stringcolor}
}

% Section title formatting
\titleformat{\section}{\large\bfseries\color{titlecolor}}{antiquefuchsia}{0em}{}
\titleformat{\subsection}{\normalsize\bfseries\color{black}}{}{0em}{}

\begin{document}

\begin{center}
    \Large \textbf{Java Programming II - Reference Sheet} \\
\end{center}

\vspace{0.2cm}
\begin{multicols}{2}

% Java Section
\section*{Java Essentials}

\subsection*{Instance Keyword}
\begin{lstlisting}
// "this" refers to the current object instance
public class Dog {
    String name;
    public Dog(String name) {
        this.name = name;
    }
}
\end{lstlisting}

\subsection*{Inheritance \& Overriding}
\begin{lstlisting}
// Method overriding with @Override
class Animal {
    void speak() { System.out.println("Sound"); }
}
class Dog extends Animal {
    @Override
    void speak() { System.out.println("Bark"); }
}
\end{lstlisting}

\subsection*{Abstract \& Interface}
\begin{lstlisting}
// Abstract class has abstract method(s)
abstract class Shape {
    abstract void draw();
}

// Interface contains only abstract methods by default
interface Drawable {
    void draw();
}
\end{lstlisting}

\subsection*{Instanceof \& synchronized}
\begin{lstlisting}
// Check if object is an instance of a class
if (obj instanceof Dog) {
    ((Dog) obj).speak();
}
synchronized(obj) { /* safe code */ }
\end{lstlisting}

\subsection*{ArrayList}
\begin{lstlisting}
// Dynamic array
ArrayList<String> list = new ArrayList<>();
list.add("Hello");
\end{lstlisting}
\subsection*{Modifiers}
public – accessible everywhere \\
private – accessible within class only \\
protected – accessible within package and subclasses
\subsection*{String Methods \& Format}
\begin{lstlisting}
// trim() removes leading/trailing spaces
String name = " John ".trim().toUpperCase();
// Format a string with variables
String s = String.format("Age: %d", 20);
\end{lstlisting}

\subsection*{Parsing}
\begin{lstlisting}
// Convert String to int or double
int x = Integer.parseInt("5");
double d = Double.parseDouble("3.14");
\end{lstlisting}



% JavaFX Section
\section*{JavaFX Essentials}

\subsection*{App Structure}
\begin{lstlisting}
// JavaFX main app structure
public class App extends Application {
  public void start(Stage s) {
    VBox root = new VBox(); // layout container
    s.setScene(new Scene(root, 300, 200));
    s.setTitle("My App");
    s.show();
  }
}
\end{lstlisting}

\subsection*{Layouts}
\begin{lstlisting}
// VBox with spacing and alignment
VBox v = new VBox(10);
v.setAlignment(Pos.CENTER);

// GridPane with gaps
GridPane g = new GridPane();
g.setHgap(10); g.setVgap(10);
\end{lstlisting}

\subsection*{TextField, Button}
\begin{lstlisting}
// TextField with prompt and disabled editing/focus
TextField tf = new TextField();
tf.setPromptText("Enter");
tf.setEditable(false);
tf.setFocusTraversable(false);

// Button with action handler
Button btn = new Button("Click");
btn.setOnAction(e -> doSomething());
\end{lstlisting}

\subsection*{ImageView}
\begin{lstlisting}
// Load and display an image
Image img = new Image("file:img.png");
ImageView iv = new ImageView(img);
iv.setFitWidth(100);
iv.setPreserveRatio(true);
\end{lstlisting}

% Animations & Media
\section*{JavaFX Animation}
\begin{lstlisting}
// Fade animation: 1s duration, fully visible to 30% opacity
FadeTransition ft = new FadeTransition(Duration.seconds(1), node);
ft.setFromValue(1.0); ft.play();
// Move node by X axis
TranslateTransition tt = new TranslateTransition(Duration.seconds(1), node);
tt.setByX(100); tt.play();
// Rotate 180 degrees
RotateTransition rt = new RotateTransition(Duration.seconds(2), node);
rt.setByAngle(180); rt.play();
// Play multiple animations together
ParallelTransition pt = new ParallelTransition(rt, ft);
pt.play();
// Scale to 2x size
ScaleTransition st = new ScaleTransition(Duration.seconds(1), node);
st.setToX(2.0); st.setToY(2.0); st.play();
// Fill color from red to blue
FillTransition fill = new FillTransition(Duration.seconds(1), shape);
fill.setFromValue(Color.RED); fill.setToValue(Color.BLUE); fill.play();
// Stroke color from black to green
StrokeTransition stroke = new StrokeTransition(Duration.seconds(1), shape);
stroke.setFromValue(Color.BLACK); stroke.setToValue(Color.GREEN); stroke.play();
// Pause for 2 seconds
PauseTransition pause = new PauseTransition(Duration.seconds(2));
pause.setOnFinished(e -> System.out.println("Paused done!")); pause.play();
// Move along a path
Path path = new Path();
path.getElements().addAll(new MoveTo(0, 0), new LineTo(100, 100));
PathTransition pathT = new PathTransition(Duration.seconds(2), path, node); pathT.play();
// Play animations one after another
SequentialTransition seq = new SequentialTransition(ft, rt); 
seq.play();
\end{lstlisting}

\subsection*{Event Handler}
\begin{lstlisting}[language=java]
button.setOnAction(e ->{
});
node.setOnKeyPressed(e ->{
});
mouse.setOnMouseClicked(e ->{
});
\end{lstlisting}

\section*{Media (Audio)}
\begin{lstlisting}
// Load and play audio file
Media m = new Media(new File("music.mp3").toURI().toString());
MediaPlayer mp = new MediaPlayer(m);
mp.play(); // Also: pause(), stop(), etc.
\end{lstlisting}

% File I/O and Networking
% File I/O
\section*{File I/O}
\begin{lstlisting}
// Read
Scanner sc = new Scanner(new File("file.txt"));
while(sc.hasNextLine()) System.out.println(sc.nextLine());
sc.close();
// Write
PrintWriter pw = new PrintWriter("file.txt");
pw.println("Text"); pw.close();
// Binary
FileOutputStream out = new FileOutputStream("file.bin");
out.write(65); out.close();
FileInputStream in = new FileInputStream("file.bin");
int byte = in.read(); in.close();
// Checks
File f = new File("file.txt");
f.exists() && f.canRead() && !f.isDirectory();
\end{lstlisting}
\subsection*{Java Primitive Types }
\begin{lstlisting}[language=Java, basicstyle=\footnotesize\ttfamily]
// Type     Size     Default     Range / Example
boolean   1 bit    false      true / false
byte      8 bit    0          -128 to 127
char     16 bit    '\u0000'   Unicode ('A', '1')
int      32 bit    0          -2B to 2B
float    32 bit    0.0f       Decimal (add f)
double   64 bit    0.0        Precise decimal
\end{lstlisting}
\subsection*{Threads}
\begin{lstlisting}
new Thread(() -> { System.out.println("Run");
    try { Thread.sleep(1000); } catch (Exception e) {}}).start();
    class MyRunnable implements Runnable {
        public void run() { System.out.println("Runnable"); }
    }
    class MyThread extends Thread {
        public void run() { System.out.println("Subclass"); }
    }
} Thread.sleep(ms);  // Pause t.join(); // Wait 
\end{lstlisting}

\section*{Networking}
\subsection*{Server}
\begin{lstlisting}
// Server waits for a client connection
ServerSocket server = new ServerSocket(6000);
Socket client = server.accept();
BufferedReader in = new BufferedReader(
  new InputStreamReader(client.getInputStream()));
\end{lstlisting}

\subsection*{Client}
\begin{lstlisting}
// Client connects and sends message
Socket s = new Socket("localhost", 6000);
PrintWriter out = new PrintWriter(s.getOutputStream(), true);
out.println("Hi Server");
\end{lstlisting}


\end{multicols}


\end{document}